% Ejemplo de documento LaTeX
% Tipo de documento y tamaño de letra
\documentclass[12pt]{article}
% Preparando para documento en Español.
% Para documento en Inglés no hay que hacer esto.
\usepackage[spanish]{babel}
\selectlanguage{spanish}
\usepackage[utf8]{inputenc}
% EL titulo, autor y fecha del documento
\title{Tutorial breve de los comandos de Bash}
\author{Paulina Valenzuela}
\date{5 de Febrero de 2015}
% Aqui comienza el cuerpo del documento
\begin{document}
% Construye el título
\maketitle
\section{¿Qué es {\tt bash}?}
Bash es un interpretador de comandos utilizado sobre el sistema operativo Linux.
Su función es de mediar entre el usuario y el sistema.
\section{A continuacion la lista de comandos}
\begin{enumerate}
\item {\tt ls} (Lista los archivos y directorios)
\item {\tt less} (Ver el contenido de archivos)
\item {\tt file} (Nos informa sobre el tipo de archivo)
\item {\tt pwd} (Nos dice en que directorio nos encontramos)
\item {\tt cd} (Cambia Directorios)
\item {\tt ls-a} (Enlista el contenido de un directorio, incluyendo archivos ocultos.)
\item {\tt man} (Busca la pagina en que se encuentra dicho comando)
\item {\tt man -k} (Hace una busqueda mientras palabras clave)
\item {\tt mkdir} (Crea un directorio)
\item {\tt rmdir} (Elimina el directorio)
\item {\tt touch} (Crea un archivo en blanco)
\item {\tt cp} (Copia un archivo o directorio)
\item {\tt mv} (Mueve a archivo o a directorio)
\item {\tt rm} (Elimina un archivo)
\item {\tt vi} (Edita un archivo)
\item {\tt cat} (Ve un archivo)
\item {\tt chmod} (Cambia los permisos en un directorio o archivo)
\item {\tt ls -ld} (Ve los archivos en un directorio especifico)
\item {\tt head} (Va ala primera linea)
\item {\tt tail} (Va ala ultima linea)
\item {\tt sort} (Organiza los datos en orden)
\item {\tt nl} (Imprime los numeros de linea antes que los datos)
\item {\tt wc} (Imprime un recuento de lineas, palabras y caracteres)
\item {\tt cut} (Corta los datos y muestra solo en la pantalla archivos especificos)
\item {\tt sed} (Hace una busqueda y remplaza)
\item {\tt uniq} (Quita las lineas duplicadas)
\item {\tt tac} (Imprime los datos en orden inverso)
\end{enumerate}

% Nunca debe faltar esta última linea.
\end{document}
\item {\tt file}
